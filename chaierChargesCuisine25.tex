\documentclass[a4paper, 12pt]{report}
\usepackage[utf8]{inputenc}
\usepackage[T1]{fontenc}
\usepackage[french]{babel}
\usepackage{graphicx}
\usepackage{hyperref}
\usepackage{array}
\usepackage{tabularx}
\usepackage{geometry}
\geometry{left=2.5cm, right=2.5cm, top=2.5cm, bottom=2.5cm}

\title{Cahier des Charges \\ \textbf{Nom du Projet}}
\author{Ma Société \\ Client : Nom du Client}
\date{\today}

\begin{document}

\maketitle

\tableofcontents

% ===== 1. Introduction =====
\chapter*{Introduction}
\addcontentsline{toc}{chapter}{Introduction}
Ce document constitue le cahier des charges du projet \textbf{Nom du Projet}. Il décrit les objectifs, les exigences fonctionnelles et techniques, ainsi que les contraintes à respecter.

% ===== 2. Contexte et Objectifs =====
\chapter{Contexte et Objectifs}
\section{Contexte}
Présentation du projet, du client et des enjeux.

\section{Objectifs}
\begin{itemize}
    \item Objectif principal
    \item Sous-objectifs (si nécessaire)
\end{itemize}

% ===== 3. Exigences Fonctionnelles =====
\chapter{Exigences Fonctionnelles}
\section{Fonctionnalités Principales}
\begin{tabularx}{\textwidth}{|l|X|}
    \hline
    \textbf{ID} & \textbf{Description} \\ \hline
    F1 & Description de la fonctionnalité 1 \\ \hline
    F2 & Description de la fonctionnalité 2 \\ \hline
\end{tabularx}

\section{Cas d'Utilisation (Optionnel)}
Décrire les scénarios d'utilisation sous forme de diagrammes ou de listes.

% ===== 4. Exigences Techniques =====
\chapter{Exigences Techniques}
\section{Environnement Technique}
\begin{itemize}
    \item Système d'exploitation
    \item Langages/frameworks
    \item Base de données
    \item Contraintes matérielles
\end{itemize}

\section{Performance et Sécurité}
\begin{itemize}
    \item Temps de réponse
    \item Normes de sécurité (RGPD, etc.)
\end{itemize}

% ===== 5. Livrables et Planning =====
\chapter{Livrables et Planning}
\section{Livrables Attendus}
Liste des documents, logiciels, rapports, etc., à fournir.

\section{Planning Prévisionnel}
\begin{tabularx}{\textwidth}{|l|X|r|}
    \hline
    \textbf{Phase} & \textbf{Description} & \textbf{Date} \\ \hline
    Phase 1 & Analyse des besoins & JJ/MM/AAAA \\ \hline
    Phase 2 & Développement & JJ/MM/AAAA \\ \hline
    Phase 3 & Tests et livraison & JJ/MM/AAAA \\ \hline
\end{tabularx}

% ===== 6. Budget =====
\chapter{Budget (Optionnel)}
\section{Coûts Estimés}
Tableau ou description des coûts (si applicable).

% ===== 7. Conclusion =====
\chapter*{Conclusion}
\addcontentsline{toc}{chapter}{Conclusion}
Résumé des points clés et validation par les parties prenantes.

\end{document}